\documentclass[a4paper]{scrartcl}
\usepackage{bogwonch}

\title{Authorization Logic for Mobile Ecosystems}
\subtitle{Third Year Report}
\author{Joseph Hallett}
\date\today

\begin{document}
\maketitle

\begin{abstract}
  This report describes the third year work of my PhD.
  I describe what I've been working on this year,
  And what I still need to do in the time remaining.
  I say what I'm going to put in my thesis.
\end{abstract}

\section{Introduction}
\label{sec:introduction}

\section{Work completed in the third year}
\label{sec:work}

\subsection{Typed AppPAL}
\label{sec:types}

As part of SecPAL's safety condition all variables in an assertions head must be
used in the body.  When trying to describe policies we found a common pattern is
to give a person who satisfied some property (for example they were a staff
member) the ability to say statements about apps.
For example, Alice might be willing to let her friends say what apps are
suitable for her children.  This could be expressed in SecPAL as follows:
\begin{lstlisting}
'alice' says Friend can-say
  App isSuitableFor(Child)
  if Friend isFriend,
     App isApp,
     Child isChild.
\end{lstlisting}
The conditionals in this assertion add unnecessary noise to the assertion. We
know from the names of the variable what set of constants might be used to used
to instantiate it (Alice's friends, apps or children). To avoid this noise I
added a sugared syntax to AppPAL that allows variables to declare their
\emph{type}.  Using the sugared notation the above statement becomes:
\begin{lstlisting}
'alice' says Friend:F can-say
  App:A isSuitableFor(Child:C).
\end{lstlisting}
\begin{figure}
  \newcommand{\nonterminal}[1]{$\langle$#1$\rangle$}
  \newcommand{\terminal}[1]{\textbf{#1}}
  \begin{tabular}{r c l}
    \footnotesize
    \nonterminal{E}         & $\coloneqq$ & \nonterminal{Variable} $\vert$ \terminal{'constant'} \\
    \nonterminal{Variable}  & $\coloneqq$ & \terminal{Type}\terminal{:}\terminal{VariableName} \\
                            & $\vert$     & \terminal{VariableName}
  \end{tabular}
  \caption{Changes to SecPAL's variable syntax.}
  \label{fig:apppal-types}
\end{figure}
The changes to SecPAL's syntax is shown is \autoref{fig:apppal-types}.
After an assertion has been parsed the variables with types are extracted for
each variable a conditional is added that \texttt{VariableName \emph{is}Type},
and the types are removed from the variables.

This gives a cleaner policy language however it also means that the predicates
used start to have some intrinsic meaning.  If a predicate starts with
\texttt{is} then it is describing some property of the predicates subject.
SecPAL did not require predicates follow any naming conventions, however with
AppPAL we have started to give predicates meaning based on their name.

\subsection{BYOD Policies}
\label{sec:byod}

\subsection{Automatic Analysis of Policies}
\label{sec:lint}

When examining an AppPAL policy it is natural to wonder whether the policy is as
optimal as it could be.  Does an assertion context contain enough statements to
use a given rule?  If there are multiple ways of deciding whether some statement
is true or not does one rule require far less statements than any other?  Does
one rule require only a subset of the facts of another rule, implying the second
is redundant?



\subsection{Plausible AppPAL}
\label{sec:plausible}

\section{Plan for thesis submission}
\label{sec:thesis}
 

\end{document}